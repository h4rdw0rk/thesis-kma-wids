\chapter*{PHỤ LỤC B: THÔNG TIN PHẦN MỀM SỬ DỤNG}
\addcontentsline{toc}{chapter}{PHỤ LỤC B}

\renewcommand{\baselinestretch}{1.3}
\begin{itemize}
\item Chi tiết phần mềm trên AP \& Sensor:
\end{itemize}

\begin{table}[H]
\centering
\small
\setlength{\extrarowheight}{1pt}
\begin{tabular}{|p{3.5cm}|p{9cm}|}
\hline
\multicolumn{1}{|c|}{\textbf{Tên phần mềm}} & \multicolumn{1}{c|}{\textbf{Phiên bản sử dụng}} \\ \hline
Firmware                                    & OpenWrt Designated Driver 50107                 \\ \hline
Kernel Linux                                & 4.4.14 mips GNU/Linux                           \\ \hline
Daemonlogger                                & 1.2.1                                           \\ \hline
Kismet Drone                                & Kismet 2013-03-R0                               \\ \hline
LuCI                                        & 17.152                                          \\ \hline
OpenVPN                                     & 2.4.2                                           \\ \hline
\end{tabular}
\end{table}

\begin{itemize}
\item Chi tiết phần mềm trên IDS Server:
\end{itemize}

\begin{table}[H]
\centering
\small
\setlength{\extrarowheight}{1pt}
\begin{tabular}{|p{3.5cm}|p{9cm}|}
\hline
\multicolumn{1}{|c|}{\textbf{Tên phần mềm}} & \multicolumn{1}{c|}{\textbf{Phiên bản sử dụng}} \\ \hline
OS                                          & Raspbian (dựa trên Debian 8.0)                  \\ \hline
Apache2                                     & 2.4.10                                          \\ \hline
Barnyard2                                   & 2.1.14                                          \\ \hline
DAQ                                         & 2.2.1                                           \\ \hline
Kernel Linux                                & 4.9.24-v7+ armv7l GNU/Linux                     \\ \hline
Kismet                                      & 2013-03-R0                               \\ \hline
Libdnet                                     & 1.12                                            \\ \hline
Libpcap                                     & 1.6.2                                           \\ \hline
Libnl                                       & 1.0.0                                           \\ \hline
Libpcre                                     & 8.35 2014-04-04                                 \\ \hline
MySQL                                       & 14.14 Distrib 5.5.54                            \\ \hline
OpenVPN                                     & 2.4.2                                           \\ \hline
Ruby                                        & 2.0.0p0                                         \\ \hline
Rails                                       & 3.2.22                                          \\ \hline
Rack                                        & 1.4.7                                           \\ \hline
Rake                                        & 0.9.6                                           \\ \hline
Sagan                                       & 1.1.8                                           \\ \hline
Snorby                                      & 2.6.3                                           \\ \hline
Snort                                       & 2.9.9.0 GRE                                     \\ \hline
\end{tabular}
\end{table}

\begin{itemize}
\item Cấu hình luật của Kismet để phát hiện các tấn công từ bên ngoài, tập tin cấu hình tại "\emph{/usr/local/etc/kismet.conf}" trên IDS Server:\\ \\
\end{itemize}

\begin{lstlisting}
# Packet filtering options:
# filter_tracker - Packets filtered from the tracker are not processed or
#                  recorded in any way.

filter_tracker=BSSID(18:A6:F7:A1:36:14) # Drone & AP

# Alerts to be reported and the throttling rates.
# alert=name,throttle/unit,burst
# See the README for a list of alert types.
alert=ADHOCCONFLICT,5/min,1/sec
alert=AIRJACKSSID,5/min,1/sec
alert=APSPOOF,10/min,1/sec
alert=BCASTDISCON,5/min,2/sec
alert=BSSTIMESTAMP,5/min,1/sec
alert=CHANCHANGE,5/min,1/sec
alert=CRYPTODROP,5/min,1/sec
alert=DISASSOCTRAFFIC,10/min,1/sec
alert=DEAUTHFLOOD,5/min,2/sec
alert=DEAUTHCODEINVALID,5/min,1/sec
alert=DISCONCODEINVALID,5/min,1/sec
alert=DHCPNAMECHANGE,5/min,1/sec
alert=DHCPOSCHANGE,5/min,1/sec
alert=DHCPCLIENTID,5/min,1/sec
alert=DHCPCONFLICT,10/min,1/sec
alert=NETSTUMBLER,5/min,1/sec
alert=LUCENTTEST,5/min,1/sec
alert=LONGSSID,5/min,1/sec
alert=MSFBCOMSSID,5/min,1/sec
alert=MSFDLINKRATE,5/min,1/sec
alert=MSFNETGEARBEACON,5/min,1/sec
alert=NULLPROBERESP,5/min,1/sec
alert=PROBENOJOIN,5/min,1/sec
alert=WPSBRUTE,5/min,1/sec

# Controls behavior of the APSPOOF alert.
apspoof=AP:ssid="KMA-WIFI",validmacs="18:A6:F7:A1:36:14"
\end{lstlisting}

\newgeometry{a4paper,left=3.5cm,right=2cm,top=3cm,bottom=2.5cm}
\headsep=14pt

\begin{itemize}
\item Cấu hình các mô-đun tiền xử lý, cũng như tập luật bổ sung cho Snort để phát hiện tấn công bên trong, tập tin cấu hình tại "\emph{/etc/snort/snort.conf}" trên IDS Server:\\
\end{itemize}

\begin{lstlisting}
# Target-Based stateful inspection/stream reassembly.  For more inforation, see README.stream5
preprocessor stream5_global: track_tcp yes, \
   track_udp yes, \
   track_icmp no, \ 
   max_tcp 262144, \
   max_udp 131072, \
   max_active_responses 2, \
   min_response_seconds 5
preprocessor stream5_tcp: policy windows, detect_anomalies, require_3whs 180, \
   overlap_limit 10, small_segments 3 bytes 150, timeout 180, \
    ports client 21 22 23 25 42 53 70 79 109 110 111 113 119 135 136 137 139 143 \
        161 445 513 514 587 593 691 1433 1521 1741 2100 3306 6070 6665 6666 6667 6668 6669 \
        7000 8181 32770 32771 32772 32773 32774 32775 32776 32777 32778 32779, \
    ports both 36 80 81 82 83 84 85 86 87 88 89 90 110 311 383 443 465 563 555 591 593 631 636 801 808 818 901 972 989 992 993 994 995 1158 1220 1414 1533 1741 1830 1942 2231 2301 2381 2578 2809 2980 3000 3001 3029 3037 3057 3128 3443 3702 4000 4343 4848 5000 5117 5250 5450 5600 5814 6080 6173 6988 7907 7000 7001 7005 7071 7144 7145 7510 7802 7770 7777 7778 7779 \
        7801 7900 7901 7902 7903 7904 7905 7906 7908 7909 7910 7911 7912 7913 7914 7915 7916 \
        7917 7918 7919 7920 8000 8001 8008 8014 8015 8020 8028 8040 8080 8081 8082 8085 8088 8090 8118 8123 8180 8181 8182 8222 8243 8280 8300 8333 8344 8400 8443 8500 8509 8787 8800 8888 8899 8983 9000 9002 9060 9080 9090 9091 9111 9290 9443 9447 9710 9788 9999 10000 11371 12601 13014 15489 19980 29991 33300 34412 34443 34444 40007 41080 44449 50000 50002 51423 53331 55252 55555 56712
preprocessor stream5_udp: timeout 180

# Portscan detection.  For more information, see README.sfportscan
preprocessor sfportscan:\
        proto { all } \
\end{lstlisting}

\begin{lstlisting}
        scan_type { all } \
        sense_level { high } \  
        logfile { portscan.log }

# SMB / DCE-RPC normalization and anomaly detection.  For more information, see README.dcerpc2
preprocessor dcerpc2: memcap 102400, events [co ]
preprocessor dcerpc2_server: default, policy WinXP, \
    detect [smb [139,445], tcp 135, udp 135, rpc-over-http-server 593], \
    autodetect [tcp 1025:, udp 1025:, rpc-over-http-server 1025:], \
    smb_max_chain 3, smb_invalid_shares ["C$", "D$", "ADMIN$"]

###################################################
# Step #7: Customize your rule set
# For more information, see Snort Manual, Writing Snort Rules
#
# NOTE: All categories are enabled in this conf file
###################################################

# site specific rules
include $RULE_PATH/local.rules
include $RULE_PATH/emerging-netbios.rules
\end{lstlisting}

\begin{itemize}
\item Đối với tấn công quét cổng, thành phần tiền xử lý sử dụng mô-đun \emph{sfportscan} có thể phát hiện hầu hết tấn công quét cổng mà không cần viết thêm luật bổ sung. Đối với tấn công khai thác lỗ hỏng MS08-067, đồ án sử dụng tập luật \emph{emerging-netbios} của Emerging Threats, có thể tải về từ trang chủ Snort. Sau đây là luật đã được Snort sử dụng trong trường hợp kiểm thử ở mục~\ref{subsection:kiem-thu-chuc-nang}:\\
\end{itemize}

\begin{lstlisting}
alert tcp any any -> $HOME_NET 445 (msg:"ET NETBIOS Microsoft Windows NETAPI Stack Overflow Inbound - MS08-067 (15)"; flow:established,to_server; content:"|1F 00|"; content:"|C8 4F 32 4B 70 16 D3 01 12 78 5A 47 BF 6E E1 88|"; content:"|00 2E 00 2E 00 5C 00 2E 00 2E 00 5C|"; reference:url,www.microsoft.com/technet/security/Bulletin/MS08-067.mspx; reference:cve,2008-4250; reference:url,www.kb.cert.org/vuls/id/827267; reference:url,doc.emergingthreats.net/bin/view/Main/2008705; classtype:attempted-admin; sid:2008705; rev:5;)
\end{lstlisting}
